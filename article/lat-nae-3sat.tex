%% File: lat-nae-3sat.tex
%% Authors: William DeMeo and Hyeyoung Shin
%% Date: Jan 11, 2015
%% Updated: Nov 15, 2016
%% Copyright (c) 2016 William DeMeo and Hyeyoung Shin
%% 
\documentclass[12pt]{amsart}
% The following \documentclass options may be useful:
% preprint      Remove this option only once the paper is in final form.
% 10pt          To set in 10-point type instead of 9-point.
% 11pt          To set in 11-point type instead of 9-point.
% numbers       To obtain numeric citation style instead of author/year.

%% \usepackage{setspace}\onehalfspacing

\usepackage{amsmath}
\usepackage{amscd,amssymb,amsthm} %, amsmath are included by default
\usepackage{latexsym,stmaryrd,mathrsfs,enumerate,scalefnt,ifthen}
\usepackage{mathtools}
\usepackage[mathcal]{euscript}
\usepackage[colorlinks=true,urlcolor=black,linkcolor=black,citecolor=black]{hyperref}
\usepackage{url}
\usepackage{scalefnt}
\usepackage{tikz}
\usepackage{color}
\usepackage[margin=1in]{geometry}
\usepackage{scrextend}

%%////////////////////////////////////////////////////////////////////////////////
%% Theorem styles
\numberwithin{equation}{section}
\theoremstyle{plain}
\newtheorem{theorem}{Theorem}[section]
\newtheorem{lemma}[theorem]{Lemma}
\newtheorem{proposition}[theorem]{Proposition}
\newtheorem{prop}[theorem]{Proposition}
\theoremstyle{definition}
\newtheorem{claim}[theorem]{Claim}
\newtheorem{corollary}[theorem]{Corollary}
\newtheorem{definition}[theorem]{Definition}
\newtheorem{notation}[theorem]{Notation}
\newtheorem{Fact}[theorem]{Fact}
\newtheorem*{fact}{Fact}
\newtheorem{example}[theorem]{Example}
\newtheorem{examples}[theorem]{Examples}
\newtheorem{exercise}{Exercise}
\newtheorem*{lem}{Lemma}
\newtheorem*{cor}{Corollary}
\newtheorem*{remark}{Remark}
\newtheorem*{remarks}{Remarks}
\newtheorem*{obs}{Observation}


%%%%%%%%%%%%%%%%%%%%%%%%%%%%%%%%%%%%%%%%
% Acronyms
%%%%%%%%%%%%%%%%%%%%%%%%%%%%%%%%%%%%%%%%
%% \usepackage[acronym, shortcuts]{glossaries}
%\usepackage[smaller]{acro}
\usepackage[smaller]{acronym}
\usepackage{xspace}

%% \acs{CSP} -- short version of the acronym\\
%% \acl{CSP} -- expanded acronym without mentioning the acronym.\\
%% \acp{CSP} -- plurals.\\
%% \acfp{CSP} -- long forms into plurals.\\
%% \acsp{CSP} -- short form into a plural.\\
%% \aclp{CSP} -- long form into a plural.\\
%% \acfi{CSP} -- Full Name acronym in italics and abbreviated form in upshape.\\
%% \acsu{CSP} -- short form of the acronym and marks it as used.\\
%% \aclu{CSP} -- Prints the long form of the acronym and marks it as used.\\

%% ------ acro defs ----------
\acrodef{ccp}[CCP]{covered coatoms problem}
\acrodef{lics}[LICS]{Logic in Computer Science}
\acrodef{sat}[SAT]{satisfiability}
\acrodef{nae}[NAE]{not-all-equal}
\acrodef{naen}[NAEN]{not-all-equal with negation}
\acrodef{ctb}[CTB]{cube term blocker}
\acrodef{tct}[TCT]{tame congruence theory}
\acrodef{wnu}[WNU]{weak near-unanimity}
\acrodef{CSP}[CSP]{constraint satisfaction problem}
\acrodef{MAS}[MAS]{minimal absorbing subuniverse}
\acrodef{MA}[MA]{minimal absorbing}
\acrodef{cib}[CIB]{commutative idempotent binar}
\acrodef{sd}[SD]{semidistributive}
\acrodef{NP}[NP]{nondeterministic polynomial time}
\acrodef{P}[P]{polynomial time}
\acrodef{PeqNP}[P $ = $ NP]{P is NP}
\acrodef{PneqNP}[P $ \neq $ NP]{P is not NP}

\newcommand{\ccp}{\acs{ccp}\xspace}
%% ------ acro macros ----------
\newcommand{\sat}{\acs{sat}\xspace}
\newcommand{\nae}{\acs{nae}\xspace}
\newcommand{\naen}{\acs{naen}\xspace}
\newcommand{\csps}{\acsp{CSP}\xspace}
\newcommand{\sd}{\acs{sd}\xspace}
\newcommand{\cib}{\acs{cib}\xspace}
\newcommand{\cibs}{\acsp{cib}\xspace}
\newcommand{\PeqNP}{\acs{PeqNP}\xspace}
\newcommand{\PneqNP}{\acs{PneqNP}\xspace}
\newcommand{\NPcomplete}{\acs{NP}-complete\xspace}
\newcommand{\NP}{\acs{NP}\xspace}
\renewcommand{\P}{\acs{P}\xspace}


%%%%%%%%%%%%%%%%%%%%%%%%%%%%%%%%%%%%%%%%%%%%%%%%%%%%%%%%%%%%%%%%%

\usepackage{inputs/proof-dashed}


%%%%%%%%%%%%%%%%%%%%%%%%%%%%%%%%%%%%%%%%%%%%%%%%%%%%%%%%%%%%%%%%%

%% Put new macros in the macros.sty file
\usepackage{inputs/macros}

\usepackage[backend=bibtex]{biblatex}
\bibliography{inputs/refs.bib}

\begin{document}

\title[NP-completeness of coatom counting]{Filter membership of coatoms
  in a\\ partition lattice is NP-complete}
\date{\today}
\author[W.~DeMeo]{William DeMeo}
\address{University of Hawaii}
\email{williamdemeo@gmail.com}

\author[H.~Shin]{Hyeyoung Shin}
\address{University of Hawaii}
\email{hyeyoungshinw@gmail.com}

%% \thanks{The authors would like to extend special thanks to...}

\maketitle

\begin{abstract}
We show that 3\sat %% [3-SAT](https://en.wikipedia.org/wiki/3sat)
reduces to the problem of deciding whether all coatoms in a certain partition
lattice are contained in the union of a collection of certain principal filters
of the lattice, so the latter problem is \NP-complete.
We conclude with a discussion of the tractability of
such filter membership problems.
\end{abstract}

\section{Introduction}
\label{sec:introduction}
We show that 3\sat %% [3-SAT](https://en.wikipedia.org/wiki/3sat)
reduces to the problem of deciding whether all coatoms in a certain partition
lattice are contained in the union of a collection of certain principal filters
of the lattice. Thus the latter problem, which we call the
\acfi{ccp}, is \NP-complete.
To prove this, we show that \nae-3\sat reduces to \ccp.
We conclude with a discussion of the tractability of
such a filter membership problem. In particular, we describe some first steps
toward a polynomial-time algorithm for solving instances of \ccp.

%% \subsection{Definitions and Notations}

\subsection{Definition of 3\sat}
Although there are other reasonable conventions, in this paper we
will take the problem 3\sat to be defined as follows:
an instance is specified by a finite set
$V = \{v_0, v_1, \dots, v_{n-1}\}$ of \emph{variables} and
a formula $\phi$ in conjunctive normal form
over $V$. To be more precise, we define a \emph{literal} to be a variable or
the negation of a variable, and we specify an instance of 3\sat 
by giving a formula $\phi$ that consists of 
a conjunction of clauses, where each clause is a 
disjunction of \emph{exactly three distinct literals}.
When $\phi$ is defined over the set $V = \{v_0, v_1, \dots, v_{n-1}\}$
of \emph{variables}, we may denote the instance
by $\phi(v_0, v_1, \dots, v_{n-1})$.
Of course, we may assume that a variable and its negation do not both
appear in the same clause, since such clauses are trivially satisfied.

A \emph{solution} to the instance $\phi$ is a function 
$f: V \to \{0,1\}$ that assigns the value $0$ or 
$1$ to each variable in such a way that the formula $\phi$ evaluates to true under
this assignment. That is, $\phi(f(v_0), f(v_1), \dots, f(v_{n-1})) = 1$.


%% \subsubsection{Example: a non-satisfiable 3SAT instance}
\begin{example}
Consider the following (non-satisfiable) instance of  3\sat:
let $V = \{v_0, v_1, v_2\}$ be the set of variables that appear in the
formula $\phi$ and let
\begin{align*}
\phi(v_0, v_1, v_2) = (v_0 \vee v_1 \vee v_2) &\wedge
(\neg v_0 \vee v_1 \vee v_2) \wedge
(v_0 \vee \neg v_1 \vee v_2) 
 \wedge (v_0 \vee v_1 \vee \neg v_2) \\
&\wedge
(\neg v_0 \vee \neg v_1 \vee v_2)
\wedge (\neg v_0 \vee v_1 \vee \neg v_2) \wedge
(v_0 \vee \neg v_1 \vee \neg v_2)
\end{align*}
A solution is a function
$f: \{v_0, v_1, v_2\} \to \{0, 1\}$, such that $\phi(f(v_0), f(v_1), f(v_2))= 1$.
It's obvious that such a function does not exist since
each of the eight possible choices for $(f(v_0), f(v_1), f(v_2))$
results in a failure of one of the clauses of $\phi$.
(If $(f(v_0), f(v_1), f(v_2)) = (0,0,0)$, then the first clause fails;
if $(1,0,0)$, then the second fails, and so on.)
\end{example}

\subsection{Definition of \nae-3\sat and \naen-3\sat}
We take the problem \nae-3\sat to be defined as follows:
an instance is specified by a finite set $V = \{v_0, v_1, \dots, v_{n-1}\}$ of
\emph{variables} and a ternary relation
$\rho \subseteq V \times V \times V$, where 
we assume
for every $(v_{i0}, v_{i1}, v_{i2}) \in \rho$ 
that $v_{i0}$, $v_{i1}$,  and $v_{i2}$ are distinct positive literals
(i.e., non-negated variables).
A \emph{solution} to such an instance %%$\rho$ %% (v_0, v_1, \dots, v_{n-1})$
is a function $f: V \to \{0,1\}$
satisfying the following:
for all $(v_{i0}, v_{i1}, v_{i2}) \in \rho$ we have
$(f(v_{i0}), f(v_{i1}), f(v_{i2})) \notin \{(0,0,0), (1,1,1)\}$.


A variation on this definition of \nae-3\sat allows
negations of variables to appear in the tuples in $\rho$.
Since this variation of the problem will be useful below, we give it a
name---\naen-3\sat (not-all-equal-with-negation 3\sat),

\begin{example}
%% \subsubsection{Example: a non-satisfiable \naen-3\sat instance}
Consider the following (non-satisfiable)
instance of \naen-3\sat:
let $V= \{v_0, v_1, v_2\}$ and define
\[
\rho(v_0, v_1, v_2) =
\{(v_0, v_1, v_2), (\neg v_0, v_1, v_2), (v_0, \neg v_1, v_2), (v_0,
v_1, \neg v_2)\}.
\]
Then $\rho(v_0, v_1, v_2)$ is not satisfiable because in order to make
it so, we would need
$(v_0, v_1, v_2) \notin \{ (0,0,0), (1,1,1)\}$ so that the
first clause is \nae, and $(v_0, v_1, v_2) \notin \{(1,0,0), (0,1,1,)\}$
so that the second clause is \nae, and 
$(v_0, v_1, v_2) \notin \{ (0,1,0), (1,0,1)\}$ so that the third
clause is \nae, and $(v_0, v_1, v_2) \notin \{(0,0,1), (1,1,0)\}$.
\end{example}

\subsection{Reduction of \naen-3\sat to \nae-3\sat}
  An instance of \naen-3\sat can be reduced to an equivalent instance of
\nae-3\sat, as we now explain.  Let $\rho(v_0, v_1, \dots, v_{n-1})$ be
an instance of \naen-3\sat over the variable set
$V = \{v_0, v_1, \dots, v_{n-1}\}$.
Introduce the variable sets $X = \{x_0, x_1, \dots, x_{n-1}\}$ and 
$Y = \{y_0, y_1, \dots, y_{n-1}\}$, where $x_i = v_i$
%% represents an occurance of $v_i$
and  $y_i = \neg v_i$.
%% represents an occurance of $\neg v_i$.
Introduce $Z = \{z, z'\}$ and $W = \{w, w', w''\}$.
Let $U = X \cup Y \cup Z \cup W$. We will define
an \nae-3\sat instance $\rho'$ over the variable set $U$ which has a
solution if and only if $\rho$ has a solution.
We define this instance as follows: let $\rho' \subseteq U \times U \times U$
and for each triple in $\rho$, we assume that the same triple is contained in $\rho'$
(but $v_i$ is denoted by $x_i$ and $\neg v_i$ is denoted by $y_i$).
Next, for each $i$ we let $(x_i, y_i, z)\in \rho'$ and
$(x_i, y_i, z')\in \rho'$. Lastly, let
\begin{equation}
  \label{eq:1}
\{(z,z',w), (z,z',w'), (z,z',w')\} \subset \rho' \quad \text{ and }
\end{equation}
\begin{equation}
  \label{eq:2}
(w,w',w'')\in \rho'.
\end{equation}
Because of (\ref{eq:2})
%% The appearance of $(w,w',w'')$ in $\rho'$
a solution to
the \naen-3\sat instance $\rho'$ assigns at least one 0 and at least one 1
to the variables in $W$.
This and (\ref{eq:1}) imply that a solution must assign distinct
values to $z$ and $z'$, and this in turn implies that,
for each $i$, the variables $x_i$ and $y_i$
must be assigned distinct values.  Thus, $y_i$ plays the role of $\neg x_i$, as desired.

\section{Reduction of 3\sat to \nae-3\sat}
In this section we recall how a 3\sat instance can be reduced to a
\nae-3\sat instance.

\vskip5mm
(This section is under major revision.)

% The problem \emph{not-all-equal satisfiability} (\nae-\sat) is a variation on \sat in which
% a formula is satisfied if and only if each clause contains at least one true
% literal and one false literal.  Thus, a clause is satisfied if and only
% if it is not the case that all literals in that clause have the same value
% (hence, ``not-all-equal'').
% The problem \nae-3\sat is the special case of \nae-\sat in which each cluase has exactly
% three literals.

% We first demostrate the well known fact that \nae-3\sat is \NP-complete.
% It is obviously in \NP because a truth assignment is a certificate
% that can be confirmed or denied in polynomial-time.
% On the other hand, \nae-3\sat is \NP-hard because 3\sat reduces to it, as we now
% verify.

% Introducd two new variables, $u$ and $v$.
% %%  3\sat to \nae-4\sat and then to reduce
% %% \nae-4\sat to \nae-3\sat. 
% Given a clause $x_1 \vee x_2 \vee x_3$ in a 3\sat formula,
% replace $x_i$ with $y_i$ and $\neg x_i$ with
% $\neg y_i$ is such a way that $x_i = T$ if and only if
% $y_i \neq u$ and $x_i = F$ if and only if $y_i = u$.
% This results in the \nae-4\sat clause 
% $(y_1, y_2, y_3, u)$.  This reduction works because
% $(y_1, y_2, y_3, u)$ is satisfied if and only if at least one of the $y_i$'s is
% different from $u$, which holds if and only if
% at least one member of $\{x_1$, $x_2$, $x_3\}$ is true.
% This shows that 3\sat is reducible to \nae-4\sat, so the latter is \NP-hard.

% To reduce \nae-4\sat to \nae-3\sat, we use our second reference variable, $v$.
% Given a clause $(y_1, y_2, y_3, u)$ in an \nae-4\sat formula,
% we use $v$ and $\neg v$ accordingly to reduce it to two 3\sat clauses
% \[(y_1, y_2, v) \wedge (y_3, u, \neg v). \] This reduction works
% because appropriate $v$ and $\neg v$ in each clause will prevent
% all-three-true or all-three-false clauses.

\section{Reduction of \nae-3\sat to \ccp}
\label{sec:reduction-nae-3sat}
Consider the relational structure $\mathbb B = \langle \{0, 1\}, R\rangle$
where $R$ is the ternary relation $\{0, 1\}^3 - \{(0,0,0), (1,1,1)\}$.
That is,
\[ R = \{(0,0,1), (0,1,0), (0,1,1), (1,0,0), (1,0,1), (1,1,0)\}.\]
The \csp associated with the relational structure $\mathbb B$
is denoted by $\operatorname{CSP}(\mathbb B)$ and described as follows:
An \defn{instance} %% of $\operatorname{CSP}(\mathbb B)$, we mean
is a (finite) relational structure $\mathbb A = \langle A, S \rangle$
with a single ternary relation $S$, and $\operatorname{CSP}(\mathbb B)$ is the
following decision problem:

%% \begin{problem}
\begin{quote}
{\bf Problem.} Given an instance $\mathbb A = \langle A, S \rangle$,
does there exist a (relational structure) homomorphism from $\mathbb A$ to $\mathbb B$?
\end{quote}
%% \end{problem}

In other words, does there exist a function
$f\colon A \rightarrow \{0,1\}$ such that $(f(a), f(b), f(c)) \in R$ whenever
$(a, b, c) \in S$?

The kernel of a function $f$ with codomain $\{0,1\}$ has two equivalence
classes---namely, $f^{-1}\{0\}$ and $f^{-1}\{1\}$.
If one of these classes is empty, then $f$ is constant 
in which case it cannot be a homomorphism into the relational structure
$\langle \{0, 1\}, R\rangle$ (since $(0,0,0)\notin R$ and $(1,1,1)\notin R$).
Therefore, the kernel of every homomorphism $f \colon \mathbb A \rightarrow \mathbb B$
has two nonempty blocks.

Now, given a partition of $A$ into two blocks, $\pi = |A_1|A_2|$,
there are exactly two functions of type $A \rightarrow \{0,1\}$
with kernel $\pi$. One is $f(x) = 0$ iff $x \in A_1$ and the other is $1-f$.
It is obvious that either both $f$ and $1-f$ are homomorphisms or neither $f$
nor $1-f$ is a homomorphism. Indeed, both are homomorphisms if and only if
for all tuples $(a,b,c) \in S$ we have $\{a,b,c\} \nsubseteq A_1$ and
$\{a,b,c\} \nsubseteq A_2$.
Neither is a homomorphism if and only if there exists
$(a,b,c) \in S$ with $\{a,b,c\} \subseteq A_1$ or
$\{a,b,c\} \subseteq A_2$.

Now, for each tuple $s = (a,b,c) \in S$, we let $\im(s)$ (or simply $\im s$)  denote the image of
$\{0,1,2\}$ under $s$ (viewing the sequence $s$ as a function with domain the 
``index set'' $\{0,1,2\}$ and codomain the set $A$).
Furthermore, we let $\< \im s\>$ denote the equivalence relation on $A$
generated by $\im s$.  Thus, if $s = (a,b,c)$, then 
\[
\< \im s \> = \{(a,b), (b,a), (a,c), (c,a), (b,c), (c,b)\} \cup \{(x,x) : x \in A\}.
\]
The partition corresponding to $\langle \im s \rangle$ is
$\pi_{\langle \im s\rangle} = |a, b, c|x_1|x_2|\cdots$.
It is clear that a function $f\colon A \rightarrow \{0, 1\}$ is a homomorphism from
$\mathbb A$ to $\mathbb B$ if and only if for all $s \in S$
the relation $\langle \im s\rangle$ does not belong to the kernel of $f$.
Therefore, a solution to the instance $\mathbb A = \langle A, S \rangle$ of
$\operatorname{CSP}(\mathbb B)$ exists if and only if there is at least one coatom
in the lattice of equivalence relations of $A$ that is not contained in the union
$ \bigcup_{s\in S} \, ^\uparrow\langle \im s \rangle $ of principal filters.

The \defn{covered coatoms problem} (\ccp) is the following:
Given a set $A$ and a list
$s_1 = (a_1, b_1, c_1), s_2=(a_2, b_2, c_2), \dots, s_n =(a_n, b_n, c_n)$
of triples with elements in $A$, decide whether all of the coatoms
of the lattice $\prod_A$ of partitions of $A$ are contained in the union
$ \bigcup_{i=1}^n \, ^\uparrow\langle \im s_i \rangle$ of principal filters.

\section{Algorithms}
We now discuss the tractability of
the filter membership problem described in Section~\ref{sec:reduction-nae-3sat}.
Specifically, we describe some first steps
toward a polynomial-time algorithm for solving instances of \ccp.
\vskip5mm
(This section is under major revision.)


%% \section{Definitions and Notations}
%% \label{sec:defin-notat}

%% My GitHub repo~\cite{overalgebras-github}.
%% \appendix
%% \section{Appendix Title}
%% This is the text of the appendix, if you need one.

%\bibliographystyle{amsplain} %% or amsalpha
%% \bibliographystyle{plain-url}
\printbibliography


\end{document}
